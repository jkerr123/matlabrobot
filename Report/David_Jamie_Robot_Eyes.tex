\documentclass[twocolumn]{article}

%%%%%%%%%%%%%%%%%%%%%%%%%%%%%%%
% STYLES - EDIT THESE WITH CARE
%%%%%%%%%%%%%%%%%%%%%%%%%%%%%%%

% Load basic packages
\usepackage{balance}  % to better equalize the last page
\usepackage{graphics} % for EPS, load graphicx instead 
\usepackage{txfonts}
\usepackage{times}    % comment if you want LaTeX's default font
\usepackage{color}
\usepackage{textcomp}
\usepackage{booktabs}
%\usepackage{ccicons}
\usepackage{todonotes}
\usepackage{float}
\usepackage{url}  
\usepackage{titling}	% allows you to move title up the page
\usepackage[pdftex]{hyperref}

% font  sizes
\usepackage{sectsty}			% set font sizes			
\sectionfont{\Large}			% (assumes default font size 10pt)
\subsectionfont{\large}
\subsubsectionfont{\large}
\paragraphfont{\normalsize}

% positioning
\setlength{\parindent}{0em}		% remove indent for new paragraph
\setlength{\parskip}{1em}		% space above paragraph
\setlength{\columnsep}{2em}		% distance between columns
\setlength{\droptitle}{-10em}

% llt: Define a global style for URLs, rather that the default one
\makeatletter
\def\url@leostyle{%
  \@ifundefined{selectfont}{\def\UrlFont{\sf}}{\def\UrlFont{\small\bf\ttfamily}}}
\makeatother
\urlstyle{leo}

\usepackage[hyphenbreaks]{breakurl}	% make URLs within one column
\usepackage[hypens]{url}
\def\UrlBreaks{\do\/\do-}

% To make various LaTeX processors do the right thing with page size.
\def\pprw{8.27in}
\def\pprh{11.69in}
\special{papersize=\pprw,\pprh}
\setlength{\paperwidth}{\pprw}
\setlength{\paperheight}{\pprh}
\setlength{\pdfpagewidth}{\pprw}
\setlength{\pdfpageheight}{\pprh}

%%%%%%%%%%%%%%%%%%%%%%%%%%%%%%%
% END OF STYLES 
%%%%%%%%%%%%%%%%%%%%%%%%%%%%%%%

% TITLE
\title{Robot Eyes Report}
\author{David Kenny, Jamie Kerr\\BSc (Hons) Applied Computing}
\date{March 2016}



%%%%%%%%%%%%%%%%%%%%%%%%%%%
% MAIN DOCUMENT STARTS HERE
%%%%%%%%%%%%%%%%%%%%%%%%%%%

% BEGIN DOCUMENT
\begin{document}

% add title
\maketitle

%%%%%%%%%%%
%%%%%%%%%%%

\begin{abstract}

The team were given the task of creating a disparity map from a pair of stereo images using techniques that were talked about in class.

\end{abstract}

%%%%%%%%%%%
%%%%%%%%%%%

\section{Introduction}
\vspace{-1ex}


%%%%%%%%%%%
%%%%%%%%%%%

\section{Background}
\vspace{-1ex}

\section{Description of Algorithm}
\vspace{-1ex}

%%%%%%%%%%%
%%%%%%%%%%%
\section{Description of Images}
\vspace{-1ex}

\section{Results and Comments}
\vspace{-1ex}

\section{Conclusion}
\vspace{-1ex}

The team had never worked with matlab before and were pleased with what they were able to achieve within this short space of time.

\section{Future Work} 
\vspace{-1ex}

If the team had more time available we would like to.  

%%%%%%%%%%%
%%%%%%%%%%%

\section*{Acknowledgements}
The team would like to thank Emanuele Trucco for his support throughout this module. 

%%%%%%%%%%%
%%%%%%%%%%%

\bibliographystyle{SIGCHI-Reference-Format}
\bibliography{RobotEyes}

% END DOCUMENT
\end{document}
